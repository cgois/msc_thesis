% TODO:
%    - Eq. numbering and better formatting.

\chapter{Quantum theory}
\thispagestyle{empty}  % Remove opening page number
\label{chap:quantum-theory}

	States, transformations and measurements are the basic building blocks in the description of any physical system. In this chapter, I review the mathematical structures that quantum theory assigns to each of these building blocks, emphasizing the aspects that most drastically differ from their classical counterparts. Pedagogical introductions from a similar viewpoint are found in \cite{nielsen_chuang_book,barnett_book,terra_amaral_baraviera_livro,wilde2013book,schumacher2010book}.

	\section{States}
	\label{sec:states}

		A structure at the core of the most unusual of quantum behaviors is \emph{entanglement} --- a property that may or not be present in composite quantum systems. The unusual correlations that entanglement may originate were pointed out as early as 1935 by Einstein, Podolsky and Rosen \cite{epr}, and also discussed by Schrödinger, who coined the term (originally, ``Verschränkung'') \cite{schrodinger_1935}. This discussion was later rescued by John Bell in 1964 \cite{bell_1964}, and since then been extensively tested and developed \cite{horodecki_2009_entanglement}, ultimately becoming a central feature of many quantum informational protocols, such as dense coding \cite{bennett_1992_superdense}, quantum key distribution \cite{bb84} and teleportation \cite{bennett_1993_teleporting}. Entanglement theory is a vast and endlessly interesting field of study in itself, some aspects of which I now review with special focus on bipartite systems, and making no attempt of comprehensiveness.

		\ornamentbreak
	
		A \emph{quantum state} is described by a \emph{density operator}, commonly denoted by $\rho$. Any density operator is a linear, unit-trace and positive semidefinite (PSD) operator\footnote{A Hermitian operator $O : \hilb \mapsto \hilb$ is said to be PSD if and only if $x^\intercal O x \geq 0, \;\forall x \in \hilb$. Equivalently, if and only if all its eigenvalues are real and nonnegative. In this case, we denote $O \succeq 0$.} in a Hilbert space $\hilb$. Any operator satisfying these properties represents a valid quantum state. Hence,
		%
		\begin{equation}
			\mathcal{D}(\hilbd{d}) = \{ \rho \in \mathcal{L}(\mathcal{H}^d) \mid \text{tr}\rho=1, \rho \succeq0 \} , 
			\label{eq:density-operator}
		\end{equation}
		%
		where $\mathcal{L}(\mathcal{H}^d)$ is the set of linear operators in $\mathcal{H}^d$, is the space of density operators in dimension $d$. Only finite-dimensional Hilbert spaces will be considered.
	
		As $\rho \succeq 0 \Longrightarrow \rho = \rho^\dagger$, we can use the spectral decomposition to write $\rho = \sum_m m \Pi_m$, where each $\Pi_m$ is a projector onto the eigenspace of $\rho$ associated with eigenvalue $m$. Such eigenvectors are orthogonal. They need not be normalized, but we can always and will take them as being. Orthogonality implies that $\Pi_m \Pi_n = \delta_{mn} \Pi_m$ and $1 \leq \text{rank}(\rho) \leq d$, and normalization that $\tr{\Pi_m} = 1$.  Furthermore, all $m \geq 0$, and $\tr{\rho} = 1 \Longrightarrow \sum_m m = 1$. 
	
		You may recall the more usual definition that a quantum state is described by a unit vector in a Hilbert space and, conversely, that such unit vectors describe quantum states. This is only true for a subset of states called \emph{pure quantum states}. Following along the tradition, we will denote pure quantum states as $\ket{\psi}$, where $\psi$ is some label that describes the state. Similar notation is used for the dual vector $\left( \ket{\psi} \right)^\dagger \equiv \bra{\psi}$, which is useful to write the inner product between any two vectors in the same space as $\braket{\psi}{\phi}$, and the outer product as $\ketbra{\psi}{\phi}$. An useful geometric intuition on these products is to interpret the inner product as the overlap between $\ket{\psi}$ and $\ket{\phi}$, and an outer product $\ketbra{\psi}{\psi}$ as a projection onto $\ket{\psi}$. Any pure state vector $\ket{\psi}$ can equivalently be described as the density operator $\rho = \ketbra{\psi}{\psi}$.
	
		Recalling that all eigenvectors of a density operator $\rho$ are normalized, in the nondegenerate case we may interpret the $\Pi_m \equiv \ketbra{m}{m}$ as projections onto the pure states labeled by $\ket{m}$, and the spectral decomposition $\rho = \sum_m m \Pi_m$ as a probability distribution, weighted by the eigenvalues $m$, over those. Whenever $\text{rank}(\rho) = 1$, we may hence infer that $\rho$ stands for a pure state. Equivalently, whenever $\rho$ is such a one-dimensional projector, the \emph{purity} $\tr{\rho^2} = 1$, while in general $1/d \leq \tr{\rho^2} \leq 1$. This is one of the reasons why density matrices are more general than pure states. All other density operators (i.e., those of non-unit rank) are said to describe \emph{mixed quantum states}. Although the spectral decomposition of $\rho$ suggests that a mixed state can be interpreted as a probability distribution over pure states, the understanding of a mixed state as lack of knowledge on the exact state of the system should not be taken literally. One of several reasons for this assertion is that there may be many pure state ensembles generating the same density operator \cite{hughston_1993_densitymatrix}.
	
		Given a basis $\{ \ket{e_i} \}_{i=1}^d$ for $\mathcal{H}^d$, any pure state $\ket{\psi}$ in $\mathcal{H}^d$ can be written as $\ket{\psi} = \sum_{i=1}^d c_i \ket{e_i}$, where the $c_i \in \mathbb{C}$ and $\sum_i \abs{c_i}^2 = 1$ due to $\braket{\psi}{\psi} = 1$. We will frequently be interested in $\mathcal{H}^2$, in which it is common to work with the orthonormal \emph{computational basis} $\{ \ket{0}, \ket{1} \}$. The vector representations associated with the computational basis elements are $\ket{0} \equiv \left( 1 \; 0 \right)^\intercal$ and $\ket{1} \equiv \left( 0 \; 1 \right)^\intercal$. Any $\ket{\psi} \in \mathcal{H}^2$ can thus be identified with $\ket{\psi} = c_1 \ket{0} + c_2 \ket{1} = \left( c_1 \; c_2 \right)^\intercal$. An extension to a generalized $d$-dimensional computational basis $\{ \ket{i} \}_{i=0}^{d-1}$ is similarly done. Due to its analogy with two-level classical systems (bits), a $\ket{\psi} \in \hilbd{2}$ is termed a quantum bit (\emph{qubit}) and, similarly, any $\ket{\psi} \in \hilbd{d}$ is a qu\emph{d}it.
	
	
		% Entanglement
		Entanglement --- and its opposite concept, \emph{separability} ---, are properties related to composite quantum systems. If we choose $2$ for the number of subsystems, the underlying Hilbert space $\hilb$ of a state $\rho$ can be correspondingly factored as $\hilb \equiv \hilb_A \otimes \hilb_B$, for choices of  $\hilb_A$ and $\hilb_B$ respecting $\text{dim} \,\hilb_A \cdot \text{dim} \,\hilb_B = \text{dim } \hilb$. Using the tensor product for composition is the quantum analogue of using the Cartesian product to build composite phase spaces in classical mechanics.
	
		Letting $\{ \ket{ \psi_i } \}_{i=1}^{d_A}$ and $\{ \ket{ \varphi_\alpha } \}_{\alpha=1}^{d_B}$ be orthonormal bases for $\hilb_A$ and $\hilb_B$, respectively, we can easily build an orthonormal basis for $\hilb$ as $\{ \ket{\psi_i} \otimes \ket{\varphi_\alpha} \}_{i=1, \alpha = 1}^{d_A, d_B}$. This means that any vector $\ket{\psi} \in \hilb$ has a decomposition
		%
		\begin{equation}
			\ket{\psi} \sum_{i=1}^{d_A}\sum_{\alpha=1}^{d_B} c_{i\alpha} \ket{\psi_i, \varphi_\alpha} ,
			\label{eq:pure-state-decomposition}
		\end{equation}
		%
		where $\ket{\psi_i, \varphi_\alpha} \equiv \ket{\psi_i} \otimes \ket{\varphi_\alpha}$, and $c_{i\alpha} \in \mathbb{C}, \,\forall i, \alpha$. Analogously, with $\{ \ketbra{\psi_i}{\psi_j} \}_{i,j=1}^{d_A}$ as a basis for $\mathcal{L}(\hilb_A)$ and $\{ \ketbra{\varphi_\alpha}{\varphi_\beta} \}_{\alpha,\beta=1}^{d_B}$ one for $\mathcal{L}(\hilb_B)$, any operator $O \in \mathcal{L}(\hilb)$ may be decomposed as
		%
		\begin{equation}
			O = \sum_{ij\alpha\beta} O_{ij\alpha\beta} \ketbra{\psi_i}{\psi_j} \otimes \ketbra{\varphi_\alpha}{\varphi_\beta} ,
			\label{eq:decomposed-operator}
		\end{equation}
		%
		with all coefficients $O_{ij\alpha\beta} \in \mathbb{C}$.

		Suppose $A \otimes \id_B \in \mathcal{L}(\hilb_A \otimes \hilb_B)$ is an operator acting trivially on $\hilb_B$. For any $O \in \mathcal{L}(\hilb_A \otimes \hilb_B)$, the expectation value of $A \otimes \id_B$ is given by
		%
		\begin{align*}
			\ev{A \otimes \id_B}
			= \tr{ \sum_{ij\alpha\beta} O_{ij\alpha\beta} \, A\ketbra{\psi_i}{\psi_j} \otimes \ketbra{\varphi_\alpha}{\varphi_\beta} } 
			= \trb{ A \left( \sum_{ij\alpha} O_{ij\alpha\alpha} \ketbra{\psi_i}{\psi_j} \right) } ,
		\end{align*}
		%
		which implies that $\trb{(A \otimes \id_B) O} = \tr{A O^A}$, where we define
		%
		$$
			O^A = \ptr{B}{O} = \sum_{ij\alpha} O_{ij\alpha\alpha} \ketbra{\psi_i}{\psi_j} .
		$$
		$O^A$ is what we call a \emph{reduced operator}, and it is related to $O$ through the \emph{partial trace} operation. This operation is especially useful when $O$ is a density operator. In that case, we say $\ptr{B}{\rho} = \rho^A$ is a \emph{reduced state}, representing the \emph{local} information contained in the subsystem in $\hilb_A$ alone. Of course, we could trace out on $\hilb_A$ instead, by writing $\rho^B = \ptr{A}{\rho}$.
	
		% Suppose $A$ is an operator acting only on the part $O^A \in \mathcal{L}(\hilb_A)$ of $O$. The corresponding operator in $\hilb$ is just $A \otimes \id_B$. To find a description for $O^A$, we notice that the expectation value $\tr{A \otimes \id_B \, O}$ should be equal to $\tr{ A \, O^A}$, and to comply with it we define the \emph{partial trace} over $B$, $\text{tr}_B : \mathcal{L}(\hilb_A \otimes \hilb_B) \mapsto \mathcal{L}(\hilb_A)$, as
		% %
		% $$
		% 	\ptr{B}{O} \equiv \sum_{ij\alpha\beta} O_{ij\alpha\beta} \,\tr{\ketbra{\psi_i}{\psi_j}} \otimes \ketbra{\varphi_\alpha}{\varphi_\beta} = \sum_{i\alpha\beta} O_{ii\alpha\beta} \ketbra{\varphi_\alpha}{\varphi_\beta}
		% $$
		% %
		% and call $O^A = \ptr{B}{O}$ the \emph{reduced} operator. This definition can be trivially adapted to tracing out $\hilb_A$ instead, and to dealing with more than two subsystems. Moreover, it can be shown that this is the unique operation satisfying the expectation value equality condition, and that it is also completely positive and trace preserving (the significance of these latter conditions will be discussed later) \cite{}\todo{}. This operation is especially useful when applied to density operators, in which case we call $\ptr{B}{\rho} = \rho^A$ the \emph{reduced state} (of $\rho$ in subsystem $A$).
	
		Given a factorization of $\hilb$, a state $\rho$ acting on $\hilb$ is said to be \emph{separable} if and only if it can be written as 
		\begin{equation}
			\rho = \sum_i p_i \rho_i^A \otimes \rho_i^B
			\label{eq:separable-state}
		\end{equation}
		where $\sum_i p_i = 1$, $p_i \geq 0, \,\forall i$, and $\rho_i^A \in \densop{\hilb_A}$; correspondingly for $\rho_i ^B$. A state that is not separable is \emph{entangled}. For pure states, this reduces to $\ket{\psi}_{AB} = \ket{\psi}_A \otimes \ket{\psi}_B$, and when this form is not possible, $\ket{\psi}_{AB}$ is entangled. 
	
		Asking whether a state $\rho \in \densop{\hilb}$ is entangled is only meaningful when the factorization structure is specified. As a matter of fact, any bipartite pure entangled state $\ket{\psi} \in \hilbd{2}_A \otimes \hilbd{2}_B$ can be made separable by a fitting choice of factorization \cite{terra_2007_singleparticle}. This observation implies that entanglement is a property of a state \emph{with respect to} a choice of subsystems, and not of the state in itself. In practice, the factorization can be imposed by the physical arrangement of the system, as is the case for spatially separated devices in correlation scenarios. Naturally, one may also discuss entanglement in larger number of subsystems \cite{horodecki_2009_entanglement}, but the complexity scales significantly fast, and the discussion would be of little usefulness to our objective.
	
		A rationale for the definition of separability comes from a preparation procedure \cite{werner_1989}. Consider two separate laboratories, each equipped with a device that prepares quantum states, and sharing a source of (classical) randomness. Given a random number $i$, generated by the source with probability $p_i$, the laboratories locally prepare states $\rho_i^A$ and $\rho_i^B$. Now suppose the first laboratory measures $\mathcal{M}_A$, and the second $\mathcal{M}_B$, each on their respective preparation. For a pair of measurement effects $E_{m_A} \in \mathcal{M}_A$ and $E_{m_B} \in \mathcal{M}_B$ (sec. \ref{sec:measurements}), we then have
		%
		$$
			p(m_A, m_B) = \sum_i p_i \tr{E_{m_A} \rho_i^A} \tr{E_{m_B} \rho_i^B} = \tr{ E_{m_A} \otimes E_{m_B} \rho} .
		$$
		%
		In the last equality, $\rho \equiv \sum_i p_i \rho_i^A \otimes \rho_i^B$ matches the definition of a separable state.
		
		With this discussion, it is also clearer that separable states are \emph{not} uncorrelated. However, they may only exhibit correlations as strong as the ones possible in classical systems, and for this reason are said to be \emph{classically correlated}. Entangled states, conversely, manifest correlations that are not classically reproducible, which makes it a intrinsically nonclassical property.
	
		Properly justified, the definition of entanglement is quite amicable. It is not, however, computationally friendly, and determining whether a given state $\rho$ can be decomposed as in eq.~\eqref{eq:separable-state} or not can be a daunting task. Even in bipartite structures, the problem is fully solved only under special circumstances, such as for pure states, dimensionally limited Hilbert spaces, or for some special families of quantum states, including Werner states and isotropic states. These will become important in due time, so we discuss them now.
	
		% Schmidt number
		For bipartite pure states of any dimension, the problem can be fully solved through the so-called Schmidt decomposition. The Schmidt decomposition theorem states that any $\ket{\psi} \in \hilb_A \otimes \hilb_B$ can be decomposed as
		%
		\begin{equation}
			\ket{\psi} = \sum_{i=1}^d \eta_i \ket{i_A} \otimes \ket{i_B} ,
			\label{eq:schmidt-decomposition}
		\end{equation}
		%
		where $\{ \ket{i_A} \}_{i=1}^{d_A}$ and $\{ \ket{i_B} \}_{i=1}^{d_B}$ are orthonormal bases for $\hilb_A$ and $\hilb_B$, respectively, $\eta_i \geq 0$, and $d = \min \{d_A, d_B\}$. We call $\eta_i$ \emph{Schmidt coefficients}, denote by $r_S(\psi)$ the number of non-zero coefficients (\emph{Schmidt rank}), and say $\{ \eta_i (\psi) \}_{i=1}^{r_S(\psi)}$ is the \emph{Schmidt spectrum}.
	
		Contrasted to eq.~\eqref{eq:pure-state-decomposition}, this is a remarkable simplification. For one, this representation requires a single sum, but also, $d$ is the \emph{minimum} local dimension, irrespective of how (finitely) large the other dimension may be. It also resembles the definition of separability versus entanglement for pure states. Actually, it can be shown that a pure bipartite state $\ket{\psi}$ is entangled if and only if its Schmidt rank $r_S(\psi)$ is greater than one; equivalently, if the Schmidt decomposition has more than one term, or if any $\eta_i = 1$, because $\sum_i \eta_i^2 = 1$.
	
		This observation urges us to ask: are some states \emph{more entangled} than others, and can a Schmidt ``something'' be used to measure this? Attempting to adequately discuss \emph{entanglement quantifiers} would be going too far. However, as some introductory basic concepts will turn useful, we discuss them with no intention on reproducing the thoroughness that can be found in \cite{plenio_2007_entanglementmeasures,terra_tese,horodecki_2009_entanglement,dagmar_2002_entanglement}.
	
		The first important thing is that there is a whole zoo of entanglement quantifiers, such as concurrence, negativity, entanglement of formation, etc. The second is that they do not always agree with each other on the ordering they impose on the set of entangled states. Thus, depending on the intended use, there may be some more adequate than others. Nevertheless, there are several desirable properties for an entanglement measure $E : \mathcal{D}(d) \mapsto \mathbb{R}$ to satisfy, such as $E(\rho) = 0$ if $\rho$ is separable, and that it should not increase under local operations with classical communication (LOCC); a condition reminiscent of, but actually weaker than the operational definition of separability given above.
	
		Making the Schmidt rank respect the first condition is easy (just subtract $1$). Using it to impose a partial ordering on the set of pure states requires some extra caution, but it can be done through majorization. We first order the \emph{Schmidt spectrum} $\{ \eta_i (\psi) \}_{i=1}^{r_S(\psi)}$ of some state $\ket{\psi}$ in non-increasing order, then define
		%
		$$
			\psi \prec \varphi \quad\Longleftrightarrow\quad \sum_{i=1}^{r} \eta_i^2(\psi) \leq \sum_{i=1}^{r} \eta_i^2(\varphi), \;\forall r .
		$$
		%
		If $\psi \prec \varphi$, we say that $\psi$ is majorized by $\varphi$, or that $\varphi$ majorizes $\psi$. In relation to entanglement, $\psi$ would then be \emph{more} entangled than $\ket{\varphi}$, in the sense that we may convert $\ket{\psi}$ to $\ket{\varphi}$ solely by means of LOCC \cite{nielsen_1999_majorization,terra_tese}. With this in mind, states for which $\eta_i = 1/\sqrt{d}$ for all $i$ are said to be \emph{maximally entangled}.
	
		Although the Schmidt decomposition only works for pure states, the idea of the Schmidt rank can be nicely generalized to an entanglement measure over mixed states. The so called \emph{Schmidt number} \cite{terhal_2000_schmidtnumber} is given by 
		%
		\begin{equation}
			r_S(\rho) = \min_{\{ \ket{\psi_i} \}_i} \{ \max_i \left[ r_S(\psi_i ) \right] \} ,
			\label{eq:schmidt-number}
		\end{equation}
		%
		where I reuse the notation $r_S$ from the Schmidt rank because the two notions are equivalent for pure states. Arguably opaque, this definition is better understood through a procedure. Starting from $\rho$, we find an ensemble of pure states $\rho = \sum_i p_i \ketbra{\psi_i}{\psi_i}$ for it, list the $r_S(\psi_i)$ for each state of the ensemble, and take the maximum. This corresponds to the inner maximization. However, the decomposition we choose for $\rho$ is not, in general, unique. So we do this procedure for all possible sets of pure states $\{ \psi_i \}_i$ that may be used to build $\rho$, and take the minimum element of the resultant set, which is what the outer minimization means. Denoting as $S_k$ the set of density operator with Schmidt number less than or equal to $k$, it will be of most importance for us that $S_{k-1} \subset S_k$, that $S_1$ is the set of separable states, that each $S_k$ is a convex set, and that its extremal points are the pure states.
	
		% PPT
		Going back to the problem of determining whether a given $\rho$ is entangled, whenever we limit the dimensions as $\hilb = \hilbd{2} \otimes \hilbd{2}$ or $\hilb = \hilbd{2} \otimes \hilbd{3}$, the \emph{positive partial transpose} (or PPT) criterion provides a necessary and sufficient condition \cite{peres_1996_ppt,horodecki_1996_ppt}. In all other cases, the condition is still sufficient, though not necessary. To understand the sufficiency affirmation, we recall that the \emph{partial transpose} is a transposition operation acting only on some subsystems. Reusing the decomposition in eq.~\eqref{eq:decomposed-operator}, the partial transpose over $B$ is defined as
		%
		$$
			O^{\intercal_B} = (\id_A \otimes T)\,O = \sum_{ij\alpha\beta} O_{ij\alpha\beta} \ketbra{\psi_i}{\psi_j} \otimes \ketbra{\varphi_\beta}{\varphi_\alpha} = \sum_{ij\alpha\beta} O_{ij\beta\alpha} \ketbra{\psi_i}{\psi_j} \otimes \ketbra{\varphi_\alpha}{\varphi_\beta} , 
		$$
		%
		where $T$ stands for the transposition map. Now suppose that we take a separable state $\rho$ and transpose, for instance, its second subsystem (the argument is equivalent for transposing the other). Then $\rho^{\intercal_B} = \sum_i p_i \rho_i^A \otimes \left( \rho_i^B \right)^\intercal$. Building on the fact that $\rho_i^B$ was a valid density operator, and that the transpose preserves its trace and eigenvalues, it follows that $\left( \rho_i^B \right)^\intercal$ is also a density operator. With $\rho_i^A$ left unchanged, this implies that $\rho^{\intercal_B} \succeq 0$. Consequently, all separable states have positive partial transpose, which also implies that if the partial transpose of some $\rho$ has negative eigenvalues, it must be entangled.
	
		Shortly after Peres made this argument \cite{peres_1996_ppt}, Horodecki et al. showed that for $d_A \cdot d_B \leq 6$ the PPT criterion is actually necessary \emph{and} sufficient: no entangled states in these factorizations have positive partial transpose \cite{horodecki_1996_ppt}. For larger dimensions, though, they also prove this is not always true; except under special circumstances. The PPT criterion is also commonly called ``Peres-Horodecki'' criterion. 
	
		% Werner states
		One such special case is that of \emph{Werner states}. They were central to the first proof that entanglement and Bell nonlocality \cite{rabelo_dissertacao,brunner_2014_nonlocality} are not equivalent concepts \cite{werner_1989}. More specifically, it was shown that a large set of entangled Werner states are nevertheless local under projective measurements. Later, the bound for locality in $d=2$ was improved several times \cite{acin_2006_grothendieck,vertesi_2008_moreefficient,hirsch_2017_betterlocalhidden}, the result was extended to POVMs \cite{barrett_2002_povmslocality}, and they were also made pivotal in the study of quantum steering \cite{wiseman_2007_steering,cavalcanti_2016_steering,uola_2020_steering} and as a test bed for the capabilities of many quantum informational protocols, such as (semi)device-independent entanglement witnesses. What makes them especially tractable is that they are highly symmetric, as Werner states are bipartite states in $\hilbd{d} \otimes \hilbd{d}$ for which $(U \otimes U) \,\rho\, (U^\dagger \otimes U^\dagger) = \rho$, where $U$ are unitary operators. It can be shown that this is a one-parameter family of states, and that they may be written as
		%
		\begin{equation}
			W_d(\alpha) = \left( \frac{d-1+\alpha}{d-1} \right) \frac{\id}{d^2} - \left( \frac{\alpha}{d-1} \right) \frac{S}{d} .
			\label{eq:werner-states}
		\end{equation}
		%
		Here, $S = \sum_{i,j=0}^{d-1} \ketbra{ij}{ji}$ is the swap operator. When written in this form, $\alpha = 0$ stands for the maximally mixed state, and $\alpha \leq 1$. Werner states are entangled if and only if $\alpha > \frac{1}{d+1}$ but, under projective measurements, they are unsteerable if and only if $\alpha \leq 1 - \frac{1}{d}$, as shown in \cite{wiseman_2007_steering}. 
	
		% Isotropic states
		A second family of states that will also make an appearance in chap. \ref{chap:pam-quantum} are the \emph{isotropic states}. Bipartite and also highly symmetric, they are defined as states in $\hilbd{d} \otimes \hilbd{d}$ for which $(U \otimes U^*) \,\rho\, (U^\dagger \otimes U^{*^\dagger}) = \rho$. They were originally constructed to aid in proofs of entanglement distillability criteria \cite{horodecki_1999_isotropic}. Later, together with Werner states, they were used to show that entanglement, EPR steering and Bell nonlocality form a strict hierarchy \cite{wiseman_2007_steering,quintino_2015_inequivalence}, and they have likewise been useful in a multitude of benchmarks. They can be described through a single real, linear parameter $\alpha$ by
		%
		\begin{equation}
			\chi(\alpha) = \left(1 - \alpha\right) \frac{\id}{d^2} + \alpha \ketbra{\Phi^+}{\Phi^+} ,
			\label{eq:isotropic-states}
		\end{equation}
		%
		with $\ket{\Phi^+} = \frac{1}{\sqrt{d}} \sum_{i = 0}^{d - 1} \ket{ii}$, a maximally entangled state. For $d = 2$, they are identical to Werner states up to local unitaries, but this is not true for larger dimensions. They are also nonseparable if and only if $\alpha > \frac{1}{d+1}$, and are unsteerable under projective measurements if and only if $\alpha \leq \frac{H_d - 1}{d-1}$, where $H_d = \sum_{n=1}^d 1/n$ is a truncated harmonic series \cite{wiseman_2007_steering}. The range of $\alpha$, for both of these families, is determined by $\rho \succeq 0$. We get valid density operators when letting $\alpha$ run in $[-\frac{1}{d^2 - 1},1]$. Particularly, the isotropic state $\chi(0)$ is the maximally mixed state, and for $\chi(1)$ we have a maximally entangled one.
	
		Maximally entangled states are resources for many informational protocols, and the performance of some protocols can be characterized through the \emph{singlet fraction}, which measures the maximal overlap of a resource $\rho$ with a maximally entangled state. Starting from $\ket{\Phi^+}$, all other maximally entangled states $\ket{\Phi}$ can be reached through local unitaries alone, $\ket{\Phi} = (U_A \otimes U_B) \ket{\Phi^+}$, thus the singlet fraction is determined by
		%
		$$
			\zeta(\rho) = \max_\Phi \braopket{\Phi}{\rho}{\Phi} .
		$$
		%
		In particular, the singlet fraction of the isotropic states is
		%
		\begin{equation}
			\zeta\left[ \chi(\alpha) \right] = \alpha + \frac{1 - \alpha}{d^2} .
			\label{eq:singlet-fraction-isotropic-states}
		\end{equation}


	%%%%%%%%%%%%%%%%%%%%%%%%%%%%%%%%%%%%%%%%%%%%%%%%%%%%%%%
	\section{Transformations}
	\label{sec:transformations}

		% Unitary transformations
		We now know how to describe static physical systems, which is something that physical systems rarely are. In introductory quantum theory, we learn that the evolution of closed quantum systems is governed by the Schrödinger equation. Its solution dictates that an initial state $\rho$ that transforms to $\rho^\prime$ does so unitarily, following $\rho^\prime = U \rho U^\dagger$. Here, $U$ must be an unitary operator, which means that $UU^\dagger = U^\dagger U = \id$. Given $U$, we can always find a Hamiltonian $H$ and a suitable interaction time to perform the evolution. Nonetheless, these are not the most general transformations that a quantum state can undergo.
		
		% CPTP
		Taking the operational approach, I will define the most general transformation as any one that takes a density operator into another, i.e., any $\mathcal{N} : \mathcal{D}(\hilb) \mapsto \mathcal{D}(\hilb^\prime)$. Then we look for what properties this implies. First of all, $\mathcal{N}$ must be linear, and the reasoning for that goes as usual: if we mix $\rho, \sigma \in \hilb$ as $w \rho + (1-w) \sigma$, then put it through $\mathcal{N}$, we surely expect the resulting state to be equivalent to independently passing $\rho$ and $\sigma$ through $\mathcal{N}$ and then mixing. Equation-wise, this means that $\mathcal{N} \left[ w \rho + (1-w) \sigma \right] = w \mathcal{N}(\rho) + (1-w) \mathcal{N}(\sigma)$. Additionally, it is easy to argue that $\mathcal{N}$ must be such that, for any $\rho \in \mathcal{D}(\hilb)$, it is trace-preserving, $\text{tr}\left[ \mathcal{N}(\rho) \right] = 1$, and positive, $\mathcal{N}(\rho) \succeq 0$.
		
		Positivity, however, is not a strong enough condition. Suppose that we add an auxiliary system $\hilb_B$ of arbitrary dimension to $\hilb$, thus $\hilb \mapsto \hilb \otimes \hilb_B$, and that $\mathcal{N} \equiv \mathcal{N}_{\hilb \rightarrow \hilb^\prime} \otimes \id_B$. In some situations like this, a positive $\mathcal{N}_{\hilb \rightarrow \hilb^\prime}$ does not guarantee that $\mathcal{N}$ will map all input states to positive operators. This is precisely the case of the transposition map previously discussed, where the fact that the partial transposition may generate non-positive operators is used as an entanglement criterion. Amending this requires the stronger condition of \emph{complete positivity} (CP), whereby any CP map $\mathcal{N}$ generates a valid density operator no matter what the dimension of $\hilb_B$ is. Together, complete positivity and trace preservation are the conditions that define a CPTP map, or \emph{quantum channel} $\mathcal{N}$, which is the most general type of quantum evolution we will consider.
		
		% Kraus
		These requirements are easily agreeable, but they are not very practical. What we must now do is find a computation-friendly representation for CPTP maps. This can be found in the Kraus representation theorem \cite{wilde_2013_book}, stating that general map $\mathcal{N} : \mathcal{L}(\hilb) \mapsto \mathcal{L}(\hilb^\prime)$ is CPTP (i.e., a quantum channel) if and only if it has a decomposition
		%
		$$
			\mathcal{N}(O) = \sum_{i=1}^D K_i O K_i^\dagger ,
		$$
		%
		with $O \in \mathcal{L}(\hilb)$, all $K_i : \mathcal{L}(\hilb) \mapsto \mathcal{L}(\hilb^\prime)$, and $\sum_{i=1}^d = K_i^\dagger K_i = \id_{\hilb}$. The limit of the sum, $D$, will be at least $1$ (in this case we recover an unitary evolution), and will never need to be larger than $\text{dim}(\hilb) \cdot \text{dim}(\hilb^\prime)$.
		
		% Choi/channel-state duality
		The Kraus representation is just one of several other convenient representations for CPTP maps \cite{wood_2015_tensor}, of which we will also need the one called \emph{Choi-matrix representation}. It comes from an application of the \emph{Choi-Jamiołkowski isomorphism} \cite{jamiolkowski,jiang_2013_channelstate}. This remarkable result states that any quantum channel $\mathcal{N} : \mathcal{L}(\hilb) \mapsto \mathcal{L}(\hilb^\prime)$ can be uniquely mapped to a bipartite state
		%
		\begin{equation}
			\rho_{\mathcal{N}} = \left(\id_{\hilb} \otimes \mathcal{N} \right) \rho_{\Phi^+} \in \mathcal{D}(\hilb) \otimes \mathcal{D}(\hilb^\prime) ,
			\label{eq:channel-to-state}
		\end{equation}
		%
		where $\rho_{\Phi^+} = \sum_{i,j=1}^{d_{\hilb}} \ketbra{ii}{jj}$ is the density operator of the previously introduced maximally entangled state. Taking a closer look, this is actually saying that the space of CPTP maps, maps to the space of bipartite quantum states, and the way to do it is to apply the $\mathcal{N}$ in question to half of that maximally entangled state, while doing nothing to the second part. Conversely, with $\rho_{\mathcal{N}}$ we can write $\mathcal{N}$'s action on some state $\rho \in \densop{\hilb}$ as
		%
		\begin{equation}
			\mathcal{N}(\rho) = \ptrb{\hilb}{\rho_{\mathcal{N}} \left( \rho^\intercal \otimes \id_{\hilb^\prime} \right)} ,
			\label{eq:state-to-channel}
		\end{equation}
		%
		showing that we can also take a bipartite state and turn it into a quantum channel. Together, eqs.~\eqref{eq:channel-to-state} and \eqref{eq:state-to-channel} consist in the so-called \emph{channel-state duality}. The facts that a positive semidefinite and unit-trace $\rho_{\mathcal{N}}$ implies in a CPTP $\mathcal{N}$ (and vice-versa) is proven in theorems 2.22 and 2.26 in \cite{watrous_book_2018}. This representation of quantum channels as bipartite states will become handy when dealing with optimization problems over channels, as it will enable us to use CPTP constraints in semidefinite programs (see secs. \ref{sec:sdp} and \ref{sec:pam-quantum-optimization}).
		

	%%%%%%%%%%%%%%%%%%%%%%%%%%%%%%%%%%%%%%%%%%%%%%%%%%%%%%%
	\section{Measurements}
	\label{sec:measurements}
	
		To finish our review of quantum theory, we must remember how to measure a quantum system. While in classical mechanics we usually gloss over the concept of measurements, taking for granted that any physical property of a state is trivially accessible, in quantum theory we must not. On a par with entanglement, measurement incompatibility is a concept at the heart of the most interesting quantum phenomena, especially those with no classical counterpart. It is unavoidably linked to some of the most intriguing consequences of quantum theory, such as Heisenberg's uncertainty principle. Many consequences also arise in correlation scenarios. Decision problems on the Einstein-Podolsky-Rosen steering scenario, for one, can be one-to-one mapped to joint measurability problems, in the sense that a measurement set is incompatible if and only if it can be used to demonstrate steering \cite{quintino_incompatibilitysteering_2014,uola_incompatibilitysteering_2014,uola_onetoonesteering_2015}. In the also widely studied Bell nonlocality scenario, generating nonlocal statistics require incompatible measurements, but not every set of incompatible measurements is sufficient to observe nonlocality \cite{quintino_2016_incompatibilitybell,quintino_2018_incompatibilitybellgeneral,bene_2018_incompatibilitybell}. In this section I review the main mathematical structures related to quantum measurements, together with some aspects of measurement incompatibility.
	
		\ornamentbreak

		Any property of a quantum system $\rho$ must be assessed by measuring it. Abstractly, a measurement procedure takes a quantum input (i.e., the state) and returns a classical output (the measurement result). Inside certain constraints, the choice of measurement for a given application depends on the property to be measured (e.g., the $\mathbf{z}$ component of a spin $1/2$ particle). The number and values of the possible outcomes are associated to this choice of measurement (e.g., either $\pm \hbar/2$ for the spin $1/2$ measurement). A quantum measurement procedure is inherently probabilistic: quantum theory goes only so far as telling us how to ascribe probabilities to each possible outcome. Continuing with the spin example, this means that the quantum formalism will only tell us the probability of getting either the $\pm \hbar/2$ result. When the measurement is actually performed, we may end up with any outcome predicted to have non-zero probability of happening. Depending on this outcome (or rather, on our knowledge of it), the very state $\rho$ that was measured is changed in a non-reversible way. This non-reversibility implies that we cannot fully access $\rho$ with a single copy of the system, a fact that is at the heart of several quantum informational protocols. Although the century-long debate on the nature of quantum measurements is certainly interesting, I will take the pragmatic route and focus on its operational definition.
		
		% POVMs
		We require a quantum measurement to be described by a set $M = \{ E_m \}_m$ of PSD operators obeying the completeness relation $\sum_m E_m = \id$. The conditions $E_m \succeq 0$ and $\sum_m E_m$ can be interpreted as enforcing that $p(m) \geq 0, \;\forall m$ and $\sum_m p(m) = 1$, for any possible $\rho$. Any such set $M$ is called a \emph{positive operator-valued measure} (or POVM; also called \emph{unsharp measurement}), and each of its elements a \emph{measurement effect}. The possible outcomes are labeled by $m$. Whenever a measurement $M$ is performed on a state $\rho$, we get result $m$ with probability $p(m) = \tr{E_m \rho}$. When $\rho = \ketbra{\psi}{\psi}$, this definition recovers the Born rule in its usual form, $p(m) = \braopket{\psi}{E_m}{\psi}$.
		
		% Projective measurements
		A special case of POVMs arise when every $E_m = \Pi_m$, and $\Pi_m \Pi_n = \Pi_m \delta_{mn}$, where the $\Pi_m$ are projection operators. It then follows that $1 \leq \abs{M} \leq d$. This is the case of \emph{projective measurements} (or PVMs; commonly also called \emph{ideal}, or \emph{sharp}, or \emph{von Neumann} measurements). Quantum mechanics courses usually introduce projective measurements through the concept of \emph{observables}, which are Hermitian operators. Recalling these can be decomposed as $A = \sum_m m \Pi_m$, we can see they define a PVM where the possible outcomes are the eigenvalues $m$ of observable $A$, and the projection operators are a set of orthonormal eigenvectors.
		
		A further restriction on measurements is sometimes put on the rank of each effect. A rank-1 projective measurement happens when $M$ is projective and $\abs{M} = d$ or, equivalently, when the associated observable $A$ has no degenerate eigenvalues.
		
		An useful intuition on POVMs is to interpret them as noisy projective measurements.\footnote{Although this an useful perspective, not all nonprojective POVMs can be interpreted in this manner. As you will see, the measurements we are about to describe are not extremal, but there are also examples of extremal POVMs which are not projective \cite{oszmaniec_2017_simulating}} Consider a sharp measurement $M^\prime = \{ \ketbra{0}{0}, \ketbra{1}{1} \}$. Performed on an arbitrary state $\rho$, we will get the result associated to the the $i$-th projection with probability $p(i) = \tr{\ketbra{i}{i} \rho}$, where $i \in \{0, 1\}$. Suppose, though, that our experimental apparatus is such that it states the wrong result with probability $(1 - w)$. Then $p(i) = w \ketbra{i}{i} + (1 - w) \ketbra{i \oplus1}{i \oplus 1}$, and a measurement describing this situation is $M = \{ w \ketbra{0}{0} + (1 - w) \ketbra{1}{1}, w \ketbra{1}{1} + (1 - w) \ketbra{0}{0} \}$. It is easy to see that $M$'s effects are not orthogonal, hence this is not a projective measurement. But it is a valid unsharp measurement.
		
		% Post measurement state
		After a measurement is performed, we may be interested in what happened to $\rho$. As already stated, performing a measurement will generally incur in a mapping $\rho \mapsto \rho^\prime$. Lüders rule states that, for a PVM $\{ \Pi_m \}_m$ returning result $m$,
		%
		\begin{equation}
			\rho^\prime = \frac{\Pi_m \rho \Pi_m}{\tr{\rho \Pi_m}} .
			\label{eq:luders-rule}
		\end{equation}
		%
		Two interesting facts are that knowing the updated state requires knowledge of the measurement result, and that retaking the same measurement will produce the same outcome with definiteness. Without access to the measurement result, but knowing that $\{ \Pi_i \}_i$ was performed, all we can say is that $\rho^\prime = \sum_m \Pi_m \rho \Pi_m$. On the other hand, POVMs results are not always reproducible, and the update rule has subtleties. While we can always decompose an effect as $E_m = K_m^\dagger K_m$, where the $K_m$ are Kraus operators, this decomposition is not generally unique. If we are able to associate $K_m$ with the underlying physical process, then it makes sense to talk of a post-measurement state, and write it as
		%
		$$
			\rho^\prime = \frac{K_m \rho K_m^\dagger}{\tr{\rho E_m}} .
		$$
		
		% Incompatibility
		One of the most intriguing aspects of the quantum measurement formalism is that it implies the existence of quantities that may not be simultaneously measured with arbitrary precision on a single copy of a quantum system. Measurements behaving in this way are called \emph{incompatible measurements}. Throughout this work, we will understand ``compatibility'' as a synonym of ``joint measurability''. A set $\mathcal{M} = \{E_m^x \}_{m, x}$ of $X$ quantum measurements indexed by $x$ with $\mathcal{O}(x)$ outcomes each is said to be \emph{jointly measurable} whenever a parent measurement $J_\ell$ exists. To be a parent measurement, $J_\ell$ must be a valid quantum measurement from which any $E_m^x \in \mathcal{M}$ may be recovered. Letting $\ell = \ell_1\ell_2\ldots\ell_X$, where each $\ell_i \in \{ 1, \ldots, \mathcal{O}(i) \}$, the latter condition requires that $\sum_\ell J_\ell \delta_{m, \ell_x} = E_m^x$. We interpret this as saying that the measurement statistics that the set $\mathcal{M}$ could generate can be reproduced by applying the single measurement $J_\ell$ then coarse-graining the results. In the plenty of situations where no parent measurement exist, $\mathcal{M}$ is called incompatible or non-jointly measurable. I will get back to this topic in sec. \ref{sec:incompatibility-robustness}, where we discuss an approach to quantify how incompatible a measurement is.
		
		Several other notions of measurement compatibility exist. Introductory quantum mechanics courses, for instance, usually discuss incompatibility as commutativity: two non-commuting observables $A$ and $B$ can only be simultaneously determined up to a degree of certainty, and a trade-off between the standard deviations on the obtained results is given by Robertson's uncertainty relation $\sigma_A \sigma_B \geq \frac{1}{2} \abs{ \langle \left[ A, B \right] \rangle }$. Notably, in the particular case of sharp measurements, commutativity is equivalent to joint measurability. Regarding POVMs, though, this is not the case, as with several other nonequivalent definitions of incompatibility, such as non-disturbance and coexistence \cite{heinosaari_2016_incompatible}.
		
		% Simulatability
		From now on, let us call $\povm(d,n)$ and $\pvm(d,n)$ the set of POVMs and projective measurements, respectively, with $n$ effects acting on $\hilbd{d}$. It is clear that $\pvm(d,n) \subset \povm(d,n)$ for any $d$ and $n$. Now suppose we want to realize an experiment in which we must perform some $M \in \povm(d,n)$ but we only have access to a subset $\mathcal{M} \subset \povm(d,n)$ where $M \notin \mathcal{M}$. Could we, somehow, reproduce the results we would obtain with $M$ by only using $\mathcal{M}$ and classical processing? In many cases, the answer is \emph{yes} \cite{guerini_tese,guerini_2017_measurementsimulability,haapasalo_2012_measurementsmixing}.
		
		As an example, in chap. \ref{chap:pam-classical} we will be interested in when some set of POVMs can be simulated using only projective measurements. A good starting point is to notice that the trivial measurement $M = \{ \id_d \}_{i=1}^n \subset \povm(d,n)$ can be simulated solely by means of classical post-processing. We can simply sample a result from the uniform distribution on $\{ 1, \ldots, n \}$. Defining the \emph{depolarizing map} as 
		%
		\begin{equation}
			\Phi_t(O) \mapsto t O + (1-t) \frac{ \tr{O} }{d} \id ,
			\label{eq:depolarizing-map}
		\end{equation}
		%
		and its action on a measurement as $\Phi_t (M) \equiv \{ \Phi_t(E_m) \}_m$, we can see that any $M \in \povm(d,n)$, when fully depolarized, is simulatable with classical randomness, hence with projective measurements. The question that now arises is whether we need to go all the way through, or if there is some $t > 0$ that suffices. Astonishingly, a non-trivial, general lower bound of $t = 1/d$ exists, and it turns the whole set $\povm(d,n)$, for any number $n$ of effects, simulatable with projective measurements \cite{oszmaniec_2017_simulating}. This bound can be improved for specific cases using optimization techniques, and it is known that, for qubits, $t = \sqrt{2/3} - \epsilon$, for some small $\epsilon$, suffices \cite{guerini_tese}.
		

	\section{Gell-Mann operators and Bloch vectors}
	\label{sec:gell-mann}

		Before ending this chapter, we must discuss an interesting representation for quantum states (and some measurement operators) in terms of vectors instead of operators. This will be central to our discussions in chap. \ref{chap:pam-classical}.

		In sec. \ref{sec:states}, we argued that any quantum state is a linear, positive semi-definite, unit-trace $d$-dimensional density operator $\rho$, and that any such $\rho$ is a quantum state. Thus we defined the set of density operators as
		%
		$$
			\mathcal{D}(d) = \{ \rho \in \mathcal{L}(\mathcal{H}^d) \mid \text{tr}\rho=1, \rho \succeq0 \} ,
		$$
		%
		where $\mathcal{L}(\mathcal{H}^d)$ is the set of linear operators in $\mathcal{H}^d$. This definition implies that $\rho^\dagger = \rho$ and $\text{tr}(\rho^2) \leq 1$, with equality only holding for pure states.

		When describing a quantum state, it is often convenient to choose some basis for $\mathcal{H}^d$ and define $\rho$ through a vector. A widely used choice of basis in $\mathcal{H}^2$ are the Pauli matrices $\{\sigma_x, \sigma_y, \sigma_z\}$ (complemented with the identity operator). The Pauli matrices are Hermitian, unitary and traceless operators that, together with the identity, form a basis in which any $2 \times 2$ Hermitian matrix can be written. Letting $v = (v_x, v_y, v_z) \in \mathbb{R}^3$ and $\sigma = (\sigma_x, \sigma_y, \sigma_z)$, it is easy to see that
		%
		$$
			\rho = \frac{\mathbf{1}_2}{2} + v \cdot \sigma
		$$
		%
		is any unit-trace Hermitian operator. Restricting to $\rho \succeq 0$ further requires that $\lvert v \rvert \leq 1$. Thus, any element of the ball $\blochballd{2} = \{ x \in \mathbb{R}^3 \mid \lvert x \vert \leq 1 \}$ is a $2$-dimensional quantum state and, conversely, any $2$-dimensional quantum state can be associated to a $3$-dimensional real vector $v$ with $\lvert v \rvert \leq 1$. Also importantly, $\lvert v \rvert = 1$ if and only if the state is pure, as can be verified by constraining $\text{tr}(\rho^2) = 1$. I will call $\blochballd{2}$ the \emph{Bloch ball}, its surface the \emph{Bloch sphere}, and any $v \in \blochballd{2}$ a \emph{Bloch vector}. This geometric interpretation of qubit states will prove to be remarkably convenient.

		Generalizing these concepts to $\mathcal{H}^d$ can be done in a similar fashion, but will lead us to an important caveat. Our starting point will be picking a basis. While there are a bunch of useful choices \cite{bertlmann_2008_bloch}, the most adequate for our future endeavors will be the generalized Gell-Mann matrices. The (standard) Gell-Mann matrices naturally extend the Pauli matrices from $\text{SU}(2)$ to $\text{SU}(3)$, and they are likewise traceless and Hermitian. As these are the most important properties we will want to preserve in our applications, it is sensible to choose the standard $\text{SU}(d)$ generators when moving on to $\mathcal{H}^d$. Apart from the identity operator, we will need another $d^2 - 1$ operators to span $\mathcal{L}(\mathcal{H}^d)$. The \emph{generalized} Gell-Mann matrices match our intents, and can be conveniently written as $\sigma = \{ \sigma^{(s)}_{jk}, \sigma^{(a)}_{jk}, \sigma^{(d)}_l \}$, where
		%
		\begin{alignat*}{3}
			&\sigma^{(s)}_{jk} = \lvert j \rangle \langle k \rvert + \lvert k \rangle \langle j \rvert, \quad &&1 \leq j < k \leq d, \\
			&\sigma^{(a)}_{jk} = -i\lvert j \rangle \langle k \rvert + i\lvert k \rangle \langle j \rvert, \quad &&1 \leq j < k \leq d, \\
			&\sigma^{(d)}_{l} = \sqrt{\frac{2}{l (l+1)}} \left( \sum_{j=1}^l \lvert j \rangle \langle j \rvert - l\lvert l + 1 \rangle \langle l + 1 \rvert \right), \quad &&1 \leq l \leq d - 1 .
		\end{alignat*}
		%

		Representing an arbitrary element of $\mathcal{L}(\mathcal{H}^d)$ asks for $d^2$ coefficients. However, because we fix $\text{tr}(\rho) = 1$, there is a redundancy that will leave us with only $d^2 - 1$ free parameters. Preserving the previous notation we then propose that
		%
		\begin{equation*}
			\rho = \frac{\mathbf{1}_d}{d} + v \cdot \sigma = \frac{\mathbf{1}_d}{d} + \sum_{i = 1}^{d^2 - 1} v_i \sigma_i
			% \label{eq:gell-mann-representation}
		\end{equation*}
		%
		where with the summation we made it explicit that now $v \in \mathbb{R}^{d^2 - 1}$, and by $\sigma_i$ we mean the elements of $\sigma$ defined above. This is, again, clearly a unit-trace and Hermitian operator. The condition $\text{tr}(\rho^2) \leq 1$ further imposes that $r_d = \lvert v \rvert \leq \sqrt{\frac{d - 1}{2d}}$, called the \emph{Bloch radius}, and convexity of $\densop{\hilb}$ implies convexity on the set of Bloch vectors. It is moreover possible, and useful, to normalize any-dimensional Bloch vectors to unity and write
		%
		\begin{equation}
			\rho = \frac{\mathbf{1}_d}{d} + r_d^{-1} v \cdot \sigma
			\label{eq:normalized-bloch-vector-representation}
		\end{equation}
		instead. This is the parameterization we will use throughout chap. \ref{chap:pam-classical}.
	
		
		Our next step would be to characterize $\blochballd{d}$ and consequently the (generalized) Bloch vectors. Unfortunately, finding a structure on $v$ that guarantees that $\rho \succeq 0$ is not as straightforward as in the $2$-dimensional case. Naively trying to associate vectors in a $d$-dimensional ball with a density operator will not take us far, as we will soon find out that many choices lead to operators having negative eigenvalues. Albeit the complete characterization of Bloch vectors for arbitrary dimensions has been done \cite{kimura_2003_bloch}, it does not lead to a natural parameterization such as in the $2$-dimensional case. We will keep calling any $\blochballd{d}$ a Bloch ``ball'', but it is important to keep in mind these sets are not actually balls, for there will in general be holes where there are no allowed Bloch vectors.

		Even being more complex than for qubits, this geometric interpretation is still fruitful. For one, its inverse map has quite an intuitive interpretation inside of quantum theory. We have so far been interested in defining which Bloch vectors lead to a density operator, but have not mentioned the converse problem --- the one of determining $v$ from $\rho$ --- at all. The answer, which can be straightforwardly verified, is that $v_i = \text{tr}(\sigma_i \rho)$. Hence, each component of $v$ is the expectation value of an observable $\sigma_i$ that can, at least in principle, be experimentally obtained. Luckily, this is the mapping we will usually worry about.

		Another useful consequence of this geometric description is that the effect of operations on a state can be interpreted visually. An specific example that will later become important is that of depolarizing channels. These channels are worst-case scenario noise models describing the situation where information on a state $\rho$ is, with some probability $p$, completely lost. Thus $\rho \mapsto (1 - p) \rho + p \frac{\mathbf{1}_d}{d}$, where $\mathbf{1}_d/d$ is the maximally mixed state (the noise). Letting $v_\rho$ be the Bloch vector associated to $\rho$, and observing that $v = 0$ defines the maximally mixed state, we are led to the conclusion that a depolarizing channel shrinks $\rho$'s Bloch vector towards the center of the Bloch sphere.

		Considering that measurement operators share similarities with density matrices, we may attempt to extend this representation. Our first observation is that not all measurement effects can be written as in eq. \ref{eq:normalized-bloch-vector-representation}, because they are not required to have unit-trace. However, as any measurement effect is a positive-semidefinite operator, those of which do have unit-trace can also be described through Bloch vectors. As already pointed out, projective measurements are an important class of measurements where every effect is a projection, which is furthermore orthogonal to all the others. For any projection $\Pi$, it is true that $\text{tr}(\Pi) = \text{rank}(\Pi)$. Therefore, all rank-1 projective measurements can be described by a set of Bloch vectors, each associated to one of its effects. As all projections with larger rank can be simulated by rank-1 projections and coarse-graining, we conclude that all projective measurements can be interpreted through Bloch vectors together with post-processing. Because, by definition, $\Pi^2 = \Pi$, then $\text{tr}(\Pi^2) = \text{tr}(\Pi) = 1$, which means that a rank-1 projection operator's Bloch vector is actually on the Bloch sphere, analogously to pure states. Finally, as a measurement's effects must sum to the identity, a nice interpretation of projective measurements on qubits arises: given one of the effect's Bloch vector, the other must be its antipodal.


		\ornamentbreak
		Quantum information is the study of the encoding, processing and decoding of information in quantum systems, which are done by preparing a state $\rho$, transforming, and measuring it. In this chapter we, learned that quantum theory tells us we should associate these processes to density operators, CPTP maps, and complete collections of positive-semidefinite operators, respectively. In many cases, we will be interested in quantities that are very nicely described by these structures, but that are nevertheless not easily computable. Luckily, several well-known and efficient optimization techniques are well suited to dealing with quantum theory. These techniques will be central to chap. \ref{chap:pam-classical} and sec. \ref{sec:pam-quantum-optimization}, so we now review them.