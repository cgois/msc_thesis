\begingroup
\let\clearpage\relax

\chapter*{Abstract}
\thispagestyle{empty}

Quantum communication is expected to become a widespread technology. Dense coding, random access coding, and quantum key distribution are some of the most outstanding communication protocols where quantum systems provide advantage over their classical counterparts. These will arguably be building blocks for the so-called quantum internet, and recent experiments prove they are feasible in practice. Even so, much of their theoretical underpinnings are still poorly understood. Prepare-and-measure scenarios --- our main theme in what is to come --- are a useful abstraction within which a common basis for these protocols can be found. By hiding most implementation details, these semi-device independent scenarios also yield interesting insights into quantum theory itself. An objective of primal importance in quantum communication tasks is determining \emph{which} quantum systems provide advantage over their classical siblings, and \emph{what} are the quantum structures that enable this to happen. Our main progress in that regard consists in providing sufficient conditions to attest a given set of quantum states --- the message carriers --- are \emph{not} advantageous. As it turns out, one of these conditions is closely linked to quantum random access coding, enabling us to prove a quantum advantage activation phenomenon for this communication task. From a more fundamental perspective, a similar criterion proves that measurement incompatibility --- a promising candidate for the origin of strictly quantum behaviors --- is insufficient to explain where this quantumness arises from. While some connections between prepare-and-measure scenarios and random access coding have already been explored in the literature, those of the former with the dense coding protocol have not. As a first step towards that end, I will provide a straightforward recipe that maps the dense coding protocol to prepare-and-measure scenarios, thereby turning them semi-device independent. Ensuingly, its analysis will provide us with entanglement witnesses, the possibility of self-testing maximally entangled states, and an optimization procedure for the dense coding protocol.


% \chapter*{Resumo}
% \thispagestyle{empty}


\endgroup