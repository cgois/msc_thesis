\begingroup
\let\clearpage\relax

\chapter*{Abstract}
\thispagestyle{empty}

Quantum communication is expected to become a widespread technology. Dense coding, random access coding, and quantum key distribution are some of the most outstanding communication protocols where quantum systems provide advantage over their classical counterparts. These will arguably be building blocks for the so-called quantum internet, and recent experiments prove they are feasible in practice. Even so, much of their theoretical underpinnings are still poorly understood. Prepare and measure scenarios --- our main theme in what is to come --- are a useful abstraction within which a common basis for these protocols can be found. By hiding most implementation details, these semi-device independent scenarios also yield interesting insights into quantum theory itself. An objective of primal importance in quantum communication tasks is determining \emph{which} quantum systems provide advantage over their classical siblings, and \emph{what} are the quantum structures that enable this to happen. Our main progress in that regard consists in providing sufficient conditions to attest a given set of quantum states --- the message carriers --- are \emph{not} advantageous. As it turns out, one of these conditions is closely linked to quantum random access coding, enabling us to prove a quantum advantage activation phenomenon for this communication task. From a more fundamental perspective, a similar criterion proves that measurement incompatibility --- a promising candidate for the origin of strictly quantum behaviors --- is insufficient to explain where this quantumness arises from. While some connections between prepare and measure scenarios and random access coding have already been explored in the literature, those of the former with the dense coding protocol have not. As a first step towards that end, I will provide a straightforward recipe that maps the dense coding protocol to prepare and measure scenarios, thereby turning them semi-device independent. Ensuingly, its analysis will provide us with entanglement witnesses, the possibility of self-testing maximally entangled states, and an optimization procedure for the dense coding protocol.


% \chapter*{Resumo}
% \thispagestyle{empty}


% Espera-se que a comunicação quântica se torne uma tecnologia amplamente difundida. A distribuição de chaves criptográficas, a codificação superdensa e os códigos de acesso aleatório são notáveis exemplos de protocolos que exibem vantagens sobre o uso de sistemas clássicos para tarefas análogas. Defensavelmente, eles servirão de base para o desenvolvimento das redes quânticas. Embora tenham sido demonstrados viáveis por uma série de experimentos recentes, ainda há muito a ser entendido acerca dos fundamentos teóricos desses protocolos. Nos cenários de preparação e medição — nosso principal tema no que segue — podemos encontrar uma base comum para todas essas tarefas. Sendo exemplos da chamada *independência de dispositivos*, eles nos permitem esconder detalhes de implementação, o que por sua vez traz a possibilidade de estudarmos a teoria quântica por si própria, ou em comparação com outras teorias físicas. Um dos objetivos básicos nas tarefas de comunicação quântica é o de determinar quais sistemas quânticos podem superar a capacidade de suas contrapartidas clássicas, e quais são as estruturas da teoria quântica por trás desse comportamento. Acerca disso, nosso principal resultado é um conjunto de condições capazes de certificar que um conjunto de preparações quânticas *não* apresenta vantagem em relação ao uso de sistemas clássicos para comunicação. Utilizando um desses critérios, evidenciaremos a existência de um fenômeno de ativação de não-classicalidade, intimamente conectado a vantagens quânticas em códigos de acesso aleatório. Através de um critério similar, também é possível demonstrar que a incompatibilidade de medições — uma das estruturas no cerne da teoria quântica — não é suficiente para explicar a origem dos comportamentos quânticos no cenário de preparação e medição. Apesar de diversas conexões entre códigos de acesso aleatório e cenários de preparação e medição já terem sido exploradas na literatura, a situação é oposta em relação à conexões com o protocolo de codificação superdensa. Em um primeiro passo nessa direção, discutiremos como ir de um cenário de codificação superdensa para um de preparação e medição, de modo a torná-los independente de dispositivos. Com isso, poderemos construir testemunhas de emaranhamento, critérios para autoteste de estados maximamente emaranhamentos, e um método para otimizar a probabilidade de sucesso em protocolos de codificação superdensa.

% Espera-se que a comunicação quântica se torne uma tecnologia amplamente difundida, tendo como parte de sua base os protocolos de distribuição de chaves criptográficas, codificação superdensa, e os códigos de acesso aleatório. Embora as vantagens do uso de sistemas quânticos nesses protocolos tenham sido demonstradas por uma série de experimentos, ainda há muito a ser entendido acerca dos seus fundamentos teóricos. Nos cenários de preparação e medição — nosso principal tema — podemos encontrar uma base comum para todas essas tarefas. Um dos objetivos básicos nas tarefas de comunicação quântica é o de determinar quais sistemas quânticos podem superar a capacidade de suas contrapartidas clássicas, e quais são as estruturas quânticas por trás desse comportamento. Acerca disso, nosso principal resultado é um conjunto de condições capazes de certificar se um conjunto de preparações quânticas não apresenta vantagem em relação ao uso de sistemas clássicos para comunicação. Utilizando um desses critérios, evidenciaremos a existência de um fenômeno de ativação de não-classicalidade, intimamente conectado a vantagens quânticas em códigos de acesso aleatório. Através de um critério similar, também demonstraremos que a incompatibilidade de medições — uma das estruturas no cerne da teoria quântica — não é suficiente para explicar a origem dos comportamentos quânticos no cenário de preparação e medição. Apesar de diversas conexões entre códigos de acesso aleatório e cenários de preparação e medição já terem sido exploradas na literatura, a situação é oposta em relação à conexões com o protocolo de codificação superdensa. Em um primeiro passo nessa direção, discutiremos como mapear um cenário de codificação superdensa para um de preparação e medição, de modo a torná-los independentes de dispositivos. Com isso, poderemos construir testemunhas de emaranhamento, critérios para autoteste de estados maximamente emaranhados, e um método para otimizar a probabilidade de sucesso em protocolos de codificação superdensa.




\endgroup