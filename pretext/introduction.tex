\chapter*{Introduction}
\addcontentsline{toc}{chapter}{Introduction}{\protect\thispagestyle{empty}}
\thispagestyle{empty}
\label{chap:introduction}

In the widest sense, \emph{quantum information} is the study of the encoding, processing and decoding of information by means of quantum systems. Its origins can be traced back to the first decades following the inception of quantum theory \cite{epr,schrodinger_entanglement_1935,bohm_hiddenvariables_1952}, and, most notably, to the now called Bell's theorem \cite{bell_1964}. These earlier discussions were largely focused on conceptual puzzles arising from the theory; a similar motivation that now grounds the field we call quantum foundations. During the 80's and 90's, groundbreaking results in quantum computing and, most notably, quantum cryptography \cite{bb84,ekert91}, connected informational with foundational aspects of quantum theory, ultimately paving the way that turned quantum information into a vast and thriving research field.

\emph{Correlation scenarios} are central to quantum information. In a correlation scenario, two or more parties locally interact with and read out information from their respective systems. Collecting the results obtained by each party, one can analyze whether they are independent or correlated. In the latter case, we can further ask \emph{to what extent} such correlations are present. When the underlying systems are assumed to be classical, these correlations obey certain bounds that, strikingly, can be violated by some quantum systems. This puts a clear divide between the behaviors of classical, quantum, and even more exotic systems \cite{prbox}. Because this disctinction is observable from measurement results alone, correlation scenarios can be used to device-independently certify quantum behaviors. In the \emph{device independent} paradigm, each system is inside a black-box to which we impose no extra assumptions. All observations happen by providing inputs to the boxes and collecting their outputs. Even if no communication is allowed between the boxes, certain quantum systems are still certifiably so. Because of this no-signalling hypothesis, any correlation manifest between the systems must have been present from the start. Remarkably, it is possible to show that quantum behaviors in this so-called Bell nonlocality scenario require \emph{entanglement} and \emph{measurement incompatibility}, which are two structures at the core of the most surprising phenomena described by quantum theory. Thus, observing nonlocal behaviors, more than showing us the boxes must contain quantum systems, also tells us these systems must be entangled, and the measurements, albeit uncharacterized, have to be incompatible. Nonlocal behaviors are precisely the characteristic that allow for provably secure quantum key distribution protocols, randomness certification, and many other informational applications of device independent correlation scenarios \cite{brunner_2014_nonlocality}. From a more foundational perspective, stashing systems into black boxes allow us to hide implementation details, and thus examine classical, quantum and other theories for themselves. The question of what, exactly, is \emph{quantum} in quantum theory has puzzled scientists for more than a century, and the device independent paradigm is closely linked to modern investigations on this topic \cite{barrett_gpts_2007,plavala_gpts_2021,spekkens_2005_contextuality}.

An important relaxation on the nonlocality scenario is to allow for signalling between devices. These (semi-)device independent communication scenarios are attracting growing attention, as particularly in the case of \emph{prepare-and-measure} scenarios. Part of this interest is due to the fact that they are useful for quantum key distribution and self-testing of quantum systems, and can be seen as building blocks for quantum communication networks. In the simplest instance of a prepare-and-measure scenario, a preparation device receives an input, then prepares and communicates \emph{some} system to a measurement device which, given another input by a second observer, outputs a result. Again, only observational data is collected by each of the two parties, which may later be collectively analyzed. Similarly to how we can tell quantum from classical behaviors apart in nonlocality scenarios, it is sometimes also possible to device independently certify that quantum --- rather than classical ---, communication happened in a prepare-and-measure experiment. A practical question arising in this context is whether some set $\mathcal{S}$ of quantum preparations can manifest nonclassical behaviors. Measurements play a crucial role in this regard. To necessarily and sufficiently certify the classicality of $\mathcal{S}$, one needs to test the preparations under all the infinitely many possible measurements. This is precisely the problem we tackle in chap. \ref{chap:pam-classical}, where we devise and a general method to certify whether an arbitrary set of preparations always behaves in a way that classical systems also could. When this is the case, $\mathcal{S}$ is not useful for quantum enhancement in communication protocols. More than constructing the method, we also use it prove the existence of a quantum advantage activation phenomenon in random access coding, which is an important communication protocol. In that same chapter, we turn the question inside-out by asking whether some set $\mathcal{M}$ of quantum \emph{measurements} is useful to reveal nonclassical behaviors. Through the method built for this second task, we prove there are incompatible measurements that nevertheless cannot give rise to nonclassical behaviors. This is in stark contrast to quantum steering scenarios \cite{wiseman_2007_steering,uola_2020_steering,cavalcanti_2016_steering}, where incompatibility is necessary and sufficient for steerability \cite{quintino_incompatibilitysteering_2014,uola_onetoonesteering_2015}.

Reaching the extreme opposite side of correlation scenarios with communication, we can also assume the devices to be fully characterized. This is exactly the case for the well-known quantum teleportation \cite{bennett_1993_teleporting} and dense coding schemes \cite{bennett_1992_superdense}. In the paradigmatic \emph{dense coding} protocol, one party communicates a $d$-dimensional quantum system to another. One can show that, by exploiting pre-existing entanglement, the receiving end may perfectly recover two classical $d$its of information, thus doubling the capacity of a noiseless classical communication channel. This celebrated protocol is also remarkable for its theoretical simplicity, and its applicability paves the way towards the development of quantum technologies. However, these protocols have only been discussed in the device \emph{dependent} case, where we have full knowledge on which states may be prepared and what measurements can be applied. In chap. \ref{chap:pam-quantum}, we propose a (semi-)device independent formulation of this protocol. Not surprisingly, this can be done using the framework of prepare-and-measure scenarios but --- now surprisingly ---, this has not been much discussed before. Our formulation of dense coding as a prepare-and-measure instance can be used to certify lower bounds on the Schmidt number of the pre-existing entangled resource, witness the entanglement of any isotropic state, and self-test maximally entangled resources, among other results therein shown. Moreover, as a first step towards the study of arbitrary quantum correlated prepare-and-measure scenarios, which are largely missing in the literature, we also study an specific case where more than one measurement choice is allowed by the measuring device.

Part \ref{part:tools}, where we review basic aspects of quantum theory, convexity and mathematical optimization, should be particularly useful to newcomers. These tools are indispensable to Part \ref{part:pam}, in which we discuss general instances of prepare and measure scenarios (chap. \ref{chap:pam}), classicality certification methods (chap. \ref{chap:pam-classical}), and the formulation of device-independent dense coding (chap. \ref{chap:pam-quantum}). Appendix \ref{ap:a} provide computational details for all our classicality results, and appendix \ref{ap:pam-quantum} proves all results presented in chap. \ref{chap:pam-quantum}.