\chapter*{Introduction}
\addcontentsline{toc}{chapter}{Introduction}
\thispagestyle{empty}
\label{chap:introduction}

In the widest sense, \emph{quantum information} is the study of the encoding, processing, and decoding of information by means of quantum systems. Its origins can be traced back to the first decades following the inception of quantum theory \cite{epr,schrodinger_entanglement_1935,bohm_hiddenvariables_1952}, and, most notably, to the now called Bell's theorem \cite{bell_1964}. These earlier discussions were largely focused on conceptual puzzles arising from the theory; a similar motivation that now grounds the field we call quantum foundations. During the '80s and '90s, groundbreaking results in quantum computing and, most notably, quantum cryptography \cite{bb84,ekert91}, connected informational with foundational aspects of quantum theory, ultimately paving the way that turned quantum information into a vast and thriving research field.

\emph{Correlation scenarios} are central to quantum information. In a correlation scenario, two or more parties locally interact with and read out information from their respective systems. When only two parties are present, it is usual to call them Alice and Bob and think of each as being in their own lab, reading out information from their own experiments. It so happens that Alice's and Bob's results can be \emph{correlated}, even in cases where no communication takes place during the experiment. There is nothing surprising in observing correlations, after all, they commonly arise in all sorts of physical systems. However, something interesting happens if we further ask \emph{to what extent} their observations are correlated: when the underlying systems are assumed to go by the laws of classical physics, the possible correlations obey a certain structure that, strikingly, can sometimes be violated by quantum systems. This puts a clear divide between the behaviors of classical, quantum, and even more exotic systems \cite{prbox}. More than that, because such violations can be determined from measurement results alone, Alice and Bob can certify whether they are dealing with quantum systems even when they have no previous knowledge about their systems. It is then usual to picture the systems as being inside black-boxes that can only be investigated by providing inputs (e.g., representing measurement choices) and collecting outputs. This is the so-called \emph{device-independent} paradigm, of which the most well-known example is that of Bell nonlocality \cite{brunner_2014_nonlocality}. Remarkably, it is possible to show that nonlocal behaviors require \emph{entanglement} and \emph{measurement incompatibility}, which are two structures at the core of the most surprising phenomena described by quantum theory. %Therefore, observing them, more than showing us the devices must contain quantum systems, also tells us these systems must be entangled, and the measurements, albeit uncharacterized, have to be incompatible.
Such behaviors are also precisely the characteristic that allows for provably secure quantum key distribution protocols \cite{ekert91}, randomness certification \cite{}, and many other informational applications of device-independent correlation scenarios \cite{brunner_2014_nonlocality}. From a more foundational perspective, stashing systems into black-boxes let us to hide implementation details, and thus examine classical, quantum, and other theories for themselves. The question of what, exactly, is \emph{quantum} in quantum theory has puzzled scientists for more than a century, and the device-independent paradigm is closely linked to modern investigations on this topic \cite{barrett_gpts_2007,plavala_gpts_2021,spekkens_2005_contextuality}.  

% Collecting the results obtained by each party, one can analyze whether they are independent or correlated. In the latter case, we can further ask \emph{to what extent} such correlations are present. When the underlying systems are assumed to be classical, these correlations obey certain bounds that, strikingly, can be violated by some quantum systems. This puts a clear divide between the behaviors of classical, quantum, and even more exotic systems \cite{prbox}. Because this distinction is observable from measurement results alone, correlation scenarios can be used to device-independently certify quantum behaviors. In the \emph{device independent} paradigm, each system is inside a black-box to which we impose no extra assumptions. All observations happen by providing inputs to the boxes and collecting their outputs. Even if no communication is allowed between the boxes, certain quantum systems are still certifiably so. Because of this no-signalling hypothesis, any correlation manifest between the systems must have been present from the start. Remarkably, it is possible to show that quantum behaviors in this so-called Bell nonlocality scenario require \emph{entanglement} and \emph{measurement incompatibility}, which are two structures at the core of the most surprising phenomena described by quantum theory. Thus, observing nonlocal behaviors, more than showing us the boxes must contain quantum systems, also tells us these systems must be entangled, and the measurements, albeit uncharacterized, have to be incompatible. Nonlocal behaviors are precisely the characteristic that allow for provably secure quantum key distribution protocols \cite{ekert91}, randomness certification, and many other informational applications of device independent correlation scenarios \cite{brunner_2014_nonlocality}. From a more foundational perspective, stashing systems into black boxes allow us to hide implementation details, and thus examine classical, quantum and other theories for themselves. The question of what, exactly, is \emph{quantum} in quantum theory has puzzled scientists for more than a century, and the device independent paradigm is closely linked to modern investigations on this topic \cite{barrett_gpts_2007,plavala_gpts_2021,spekkens_2005_contextuality}.

This thesis will be largely concerned with employing the device-independent paradigm to investigate quantum communication protocols, which are clever ways of encoding, transmitting, and subsequently decoding information using quantum systems. To be useful in practice, quantum communication must provide some kind of advantage over the common practice of using classical systems for similar tasks. This is exactly the case in the dense coding protocol \cite{bennett_1992_superdense}. In it, one party communicates a $d$-dimensional quantum system to another and, by exploiting pre-existing entanglement, the receiving end may perfectly recover two classical $d$its of information, thus doubling the capacity of a noiseless classical communication channel. This celebrated protocol is also remarkable for its theoretical simplicity, and its applicability paves the way towards the development of quantum technologies. However, dense coding has only been discussed in the device-\emph{dependent} case, where one has full knowledge on which states may be prepared and what measurements can be applied. To turn it device \emph{in}dependent, we must look for a correlation scenario that supposes communication --- the aforementioned Bell nonlocality scenario does not ---, and the natural choice is the \emph{prepare and measure scenario}. In chap. \ref{chap:pam-quantum}, we will see how doing so leads to several entanglement witnesses, the possibility of self-test maximally entangled states, and a procedure to search for which states or measurements lead to the best probability of success in the dense coding protocol.

The prepare and measure scenario (chap. \ref{chap:pam-classical}) has been attracting growing attention in recent years. Part of this interest comes from it being useful for quantum key distribution \cite{pawlowski_pamqkd_2011} and self-testing quantum systems \cite{tavakoli_selftesting_2018,tavakoli_selftesting_2020}, while they can also be seen as building blocks for quantum communication networks \cite{poderini_pamcriteria_2020,bowles_pamnetworks_2015}. In the simplest instance of a prepare and measure scenario (fig. \ref{fig:pam-scenario}), a preparation device receives an input, then prepares and communicates \emph{some} system to a measurement device which, given another input by a second observer, outputs a result. Calling our friends for help, we may picture Alice operating the preparation device, and Bob the measurement one. Many practical tasks in this scenario involve Alice wanting to send some message to Bob through a \emph{limited} amount of communication, and in that case, we can see the preparation as the \emph{encoding}, and the measurement as the \emph{decoding} of the message. Only observational data is collected by each of the two parties, which means the devices are black-boxes. Similar to how we can tell quantum from classical behaviors apart in Bell nonlocality scenarios, it is sometimes also possible to device independently certify that quantum --- rather than classical ---, communication happened in a prepare and measure experiment. As nonclassical behaviors are the ones that may lead to quantum advantages in communication, an important question arising in this context is whether some set $\mathcal{S}$ of possible quantum preparations can manifest nonclassical behaviors. Measurements play a crucial role in this regard. To necessarily and sufficiently certify the classicality of $\mathcal{S}$, one needs to test the preparations under all the infinitely many possible measurements. This is the problem we tackle in chap. \ref{chap:pam-classical}, where we devise a general method to certify whether an arbitrary set of preparations always behaves in a way that classical systems also could. When this is the case, $\mathcal{S}$ is not useful for quantum enhancement in communication protocols. More than constructing the method, we also use it to prove the existence of a quantum advantage activation phenomenon in random access coding \cite{ambainis_qracsoriginal_1999}, another important communication protocol. In that same chapter, we turn the question inside-out by asking whether some set $\mathcal{M}$ of quantum \emph{measurements} is useful to reveal nonclassical behaviors. Quantum measurements differ from their classical counterparts in that they can be \emph{incompatible}. Given that no entanglement is present in this instance of prepare and measure scenarios, incompatibility is a good candidate for the origin of quantum behaviors. Through the method built for this second task, we prove there are incompatible measurements that nevertheless cannot give rise to nonclassical behaviors, thence that incompatibility is not sufficient for nonclassicality in the prepare and measure scenario. %This is in stark contrast to quantum steering scenarios \cite{wiseman_2007_steering,uola_2020_steering,cavalcanti_2016_steering}, where incompatibility is necessary and sufficient for steerability \cite{quintino_incompatibilitysteering_2014,uola_onetoonesteering_2015}.



% An important relaxation on the nonlocality scenario is to allow for signalling between devices. These (semi-)device independent communication scenarios are attracting growing attention, as particularly in the case of \emph{prepare-and-measure} scenarios. Part of this interest is due to the fact that they are useful for quantum key distribution \cite{pawlowski_pamqkd_2011} and self-testing quantum systems \cite{tavakoli_selftesting_2018,tavakoli_selftesting_2020}, and can be seen as building blocks for quantum communication networks \cite{poderini_pamcriteria_2020,bowles_pamnetworks_2015}. In the simplest instance of a prepare-and-measure scenario, a preparation device receives an input, then prepares and communicates \emph{some} system to a measurement device which, given another input by a second observer, outputs a result. Again, only observational data is collected by each of the two parties, which may later be collectively analyzed. Similarly to how we can tell quantum from classical behaviors apart in nonlocality scenarios, it is sometimes also possible to device independently certify that quantum --- rather than classical ---, communication happened in a prepare-and-measure experiment. A practical question arising in this context is whether some set $\mathcal{S}$ of quantum preparations can manifest nonclassical behaviors. Measurements play a crucial role in this regard. To necessarily and sufficiently certify the classicality of $\mathcal{S}$, one needs to test the preparations under all the infinitely many possible measurements. This is precisely the problem we tackle in chap. \ref{chap:pam-classical}, where we devise a general method to certify whether an arbitrary set of preparations always behaves in a way that classical systems also could. When this is the case, $\mathcal{S}$ is not useful for quantum enhancement in communication protocols. More than constructing the method, we also use it prove the existence of a quantum advantage activation phenomenon in random access coding \cite{ambainis_qracsoriginal_1999}, which is an important communication protocol. In that same chapter, we turn the question inside-out by asking whether some set $\mathcal{M}$ of quantum \emph{measurements} is useful to reveal nonclassical behaviors. Through the method built for this second task, we prove there are incompatible measurements that nevertheless cannot give rise to nonclassical behaviors. This is in stark contrast to quantum steering scenarios \cite{wiseman_2007_steering,uola_2020_steering,cavalcanti_2016_steering}, where incompatibility is necessary and sufficient for steerability \cite{quintino_incompatibilitysteering_2014,uola_onetoonesteering_2015}.

% Reaching the extreme opposite side of correlation scenarios with communication, we can also assume the devices to be fully characterized. This is exactly the case for the well-known quantum teleportation \cite{bennett_1993_teleporting} and dense coding schemes \cite{bennett_1992_superdense}. In the paradigmatic \emph{dense coding} protocol, one party communicates a $d$-dimensional quantum system to another. One can show that, by exploiting pre-existing entanglement, the receiving end may perfectly recover two classical $d$its of information, thus doubling the capacity of a noiseless classical communication channel. This celebrated protocol is also remarkable for its theoretical simplicity, and its applicability paves the way towards the development of quantum technologies. However, these protocols have only been discussed in the device \emph{dependent} case, where we have full knowledge on which states may be prepared and what measurements can be applied. In chap. \ref{chap:pam-quantum}, we propose a (semi-)device independent formulation of this protocol. Not surprisingly, this can be done using the framework of prepare-and-measure scenarios but --- now surprisingly ---, this has not been much discussed before. Our formulation of dense coding as a prepare-and-measure instance can be used to certify lower bounds on the Schmidt number of the pre-existing entangled resource, witness the entanglement of any isotropic state, and self-test maximally entangled resources, among other results therein shown. Moreover, as a first step towards the study of arbitrary quantum correlated prepare-and-measure scenarios, which are largely missing in the literature, we also study an specific case where more than one measurement choice is allowed by the measuring device.

Part \ref{part:tools}, where I review basic aspects of quantum theory, convexity, and mathematical optimization, provide the theoretical foundation for later discussions. The first chapter presupposes reasonable familiarity with quantum theory. For newcomers, pedagogical references are provided therein. The tools presented in Part \ref{part:tools} will set up the stage for Part \ref{part:pam}, in which we will discuss general instances of prepare and measure scenarios (chap. \ref{chap:pam}), classicality certification methods (chap. \ref{chap:pam-classical}), and the formulation of device-independent dense coding (chap. \ref{chap:pam-quantum}). The tourist reader may want to start from chap. \ref{chap:pam} before deciding whether to read the remaining chapters. Appendix \ref{ap:a} provide computational details for the classicality results, and appendix \ref{ap:pam-quantum} proves all results presented in chap. \ref{chap:pam-quantum}.